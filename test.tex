===
difficulty = 1
topic = 2
The question goes here. It should be in LaTeX form. It doesn't matter if the que-
stion is too long. You can use
\begin{enumerate}
    \item any blocks as you usually would
    \item and math notation too, like 4^3 = 64
\end{enumerate}
===
===
difficulty = 1
topic = 2
The question goes here. It should be in LaTeX form. It doesn't matter if the que-
stion is too long. You can use
\begin{enumerate}
    \item any blocks as you usually would
    \item and math notation too, like 4^3 = 64
\end{enumerate}
===
===
difficulty = 1
topic = 2
The question goes here. It should be in LaTeX form. It doesn't matter if the que-
stion is too long. You can use
\begin{enumerate}
    \item any blocks as you usually would
    \item and math notation too, like 4^3 = 64
\end{enumerate}
===
===
difficulty = 1
topic = 2
The question goes here. It should be in LaTeX form. It doesn't matter if the que-
stion is too long. You can use
\begin{enumerate}
    \item any blocks as you usually would
    \item and math notation too, like 4^3 = 64
\end{enumerate}
===
===
difficulty = 1
topic = 2
The question goes here. It should be in LaTeX form. It doesn't matter if the que-
stion is too long. You can use
\begin{enumerate}
    \item any blocks as you usually would
    \item and math notation too, like 4^3 = 64
\end{enumerate}
===
===
difficulty = 1
topic = 2
The question goes here. It should be in LaTeX form. It doesn't matter if the que-
stion is too long. You can use
\begin{enumerate}
    \item any blocks as you usually would
    \item and math notation too, like 4^3 = 64
\end{enumerate}
===
===
difficulty = 1
topic = 2
The question goes here. It should be in LaTeX form. It doesn't matter if the que-
stion is too long. You can use
\begin{enumerate}
    \item any blocks as you usually would
    \item and math notation too, like 4^3 = 64
\end{enumerate}
===
===
difficulty = 1
topic = 2
The question goes here. It should be in LaTeX form. It doesn't matter if the que-
stion is too long. You can use
\begin{enumerate}
    \item any blocks as you usually would
    \item and math notation too, like 4^3 = 64
\end{enumerate}
===
===
difficulty = 1
topic = 2
The question goes here. It should be in LaTeX form. It doesn't matter if the que-
stion is too long. You can use
\begin{enumerate}
    \item any blocks as you usually would
    \item and math notation too, like 4^3 = 64
\end{enumerate}
===
===
difficulty = 1
topic = 2
The question goes here. It should be in LaTeX form. It doesn't matter if the que-
stion is too long. You can use
\begin{enumerate}
    \item any blocks as you usually would
    \item and math notation too, like 4^3 = 64
\end{enumerate}
===
===
difficulty = 1
topic = 2
The question goes here. It should be in LaTeX form. It doesn't matter if the que-
stion is too long. You can use
\begin{enumerate}
    \item any blocks as you usually would
    \item and math notation too, like 4^3 = 64
\end{enumerate}
===
===
difficulty = 1
topic = 2
The question goes here. It should be in LaTeX form. It doesn't matter if the que-
stion is too long. You can use
\begin{enumerate}
    \item any blocks as you usually would
    \item and math notation too, like 4^3 = 64
\end{enumerate}
===
===
difficulty = 1
topic = 2
The question goes here. It should be in LaTeX form. It doesn't matter if the que-
stion is too long. You can use
\begin{enumerate}
    \item any blocks as you usually would
    \item and math notation too, like 4^3 = 64
\end{enumerate}
===
===
difficulty = 1
topic = 2
The question goes here. It should be in LaTeX form. It doesn't matter if the que-
stion is too long. You can use
\begin{enumerate}
    \item any blocks as you usually would
    \item and math notation too, like 4^3 = 64
\end{enumerate}
===
